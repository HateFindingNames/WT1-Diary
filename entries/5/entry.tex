Logbuch, welche Bedeutung haben Feiertage wenn die Kontraste des Alltages dem Zeitgeist zum Opfer fällt?
Alles gut, ich muss mich nur wieder ein wenig sortieren ;)

Was gibt es neues aus der Welt der Werkstofftechnik? Die vergangene Session befasste sich unter anderem mit Mischbarkeit von Stoffen.
Hier ist zwar explizit die Rede von Mischbarkeit unter Metallen (Stahl!elf), die Thematik erinnert aber
auch stark an Chemie 1 und ihren Themenkomplex der \textit{Löslichkeit von Stoffen}.\par
Einleitend werden drei Fälle der Mischbarkeit von Stoffen in Schmelze und im festen Zustand unterschieden:

\begin{enumerate}
    \item Fast vollständige Nichtmischbarkeit flüssig wie kristallin (hier synonym zu fest?).
    \item Vollständige Mischbarkeit flüssig wie fest (wieder, syn. zu kristallin?).
    \item Vollständige Mischbarkeit im flüssigen und teilweise Mischbarkeit im festen.
\end{enumerate}

Der erste Punkt hat im Kontext der LV zur Konsequenz, dass er nur mit einer Folie bedacht wurde. Es wird
ein Zustandsdiagramm exemplarisch an der Mischung Fe-Pb gezeigt in dem zwei voneinander verschiedene,
parallele Linien eingezeichnet sind, die die Liquidi der jeweiligen Komponenten markieren. Ein
Zustandsdiagramm mit solcher Gestalt lässt unmittelbar die Unvereinbarkeit der betroffenen Elemente
unterhalb der Gasphase erkennen.\par
Breiter angelegt ist der zweite Fall. Folie 85 zeigt ein Zustandsdiagramm, dessen Gestalt, gemessen an
dem Fehlen konkreter Stoffbezeichnungen, wohl auf alle Mischungen mit dieser Eigenschaft verallgemeinerbar
sein wird. Hier fällt sofort auf, dass im Vergleich zum ersten Fall die obere Linie durch zwei Linien
ersetzt zu sein scheint, die sich in einem gemeinsamen - dem eutektischen (sinngemäß: \textit{eu} = gut und \textit{tektik} = schmelzen. Hervorragendes Wort) - Punkt
auf der unteren Linie treffen. Die oberen markieren hier die Liquidi der Stoffmischungen. Die untere markiert
den gemeinsamen Solidus. In den Metabereichen sind Teile einer der Komponenten noch im kristallinen Zustand.
Befindet man sich in den Metabereichen und möchte das Verhältnis der Mischkristalle zur übrigen Schmelze
wissen lässt sich das geometrisch ermittel. Vom POI zieht mein eine zur Ordinate parallele Linie. Die
relativen Streckenverhältnisse vom POI zu den nächstliegenden Schnittpunkten sind direkt in
Kristall-Schmelze-Masseverhältnisse übersetzbar.\par
Hat man es nun letztlich mit einer Kombination zu tun, die im flüssigen Mischbar ist jedoch nur teilweise
im festen bilden sich in Bereichen ausreichend geringer Konzentration einer der Komponenten und unterhalb des Solidus
(wie sich das in Suden aus mehr als zwei Komponenten Verhält werden wir hoffentlich noch sehen) Segregate.
Ich bin mir nicht sicher ob ich das Wort hier vernünftig benutzt habe - Dotierung hat mir an der Stelle besser
gefallen. Die Kristallstruktur wird von der Komponente mit der höchsten Konzentration dominiert während sich
innerhalb der Kristallstruktur verteilt noch Atome der anderen Komponente befinden. Entsprechend besitzt deren
Zustandsdiagramm an den beiden Randbereichen jeweils ein weiteres Areal.\par
Als Bonus gibt es hier noch die Situation zweier Stoffe auf die Fall drei zutrifft und deren Schmelzpunkte
ausreichend weit auseinander liegen. Beim Abkühlen aus der Mischkristallphase in Richtung Konodus kommt es
zu peritektischen Reaktionen (sinngemäß: \textit{peri} = um und \textit{tektisch} = immer noch schmelzen).
Die Schmelze umgibt (daher wohl der Name) die sich bereits ausgebildeten \(\alpha\)-Mischkristalle und reagiert sie
zu \(\beta\)-Mischkristallen um.\par\medskip

Mechanische Beanspruchung! Ich und alles andere auch kann durch mechanische Beanspruchung einen Körper
Verformen. Springy (reversibel), knetig (irreversibel) und bis zum Bruch. Wird die mechanische Spannung über
die Dehnung aufgetragen markiert der Maximalwert der mech. Spannung die \textit{Festigkeit} \(R_m\). Der Punkt
an dem es zum Bruch kommt die \textit{Duktilität}.\par
Folie 98 gibt ein wenig Aufschluss über die Hintergründe der springyness etwa eines Gummibandes. Strecken der
Molekülketten durch mechanischen Zug vermindert die Entropie des Systems Gummiband. Entropie möchte aber zu nehmen.
Hier geschieht das nicht durch Materietransport also kann es nur über Wärme geschehen. Das hat mich zu einem
kleinen Versuch verleitet der erstaunlich gut funktioniert. Strecke ich den Dichtgummi eines Einmachglases
kann ich fühlen, wie es sich deutlich erwärmt (zugegebenermaßen muss ich es mir an die Lippen halten um einen
Unterschied zu merken). Wirklich spannend wird es aber wenn ich es dehne, im gedehnten Zustand abkühle (drauf pusten)
und dann zusammen ziehen lasse. Es wird deutlich kühler als die Umgebung. Offenbart sich mir hier das Kernstück
eines Low-Tec Kühlschrankes? 0o\par
Zurück zu den Metallen: ursächlich für die mehr oder weniger gute Verformbarkeit, Unterschiede in Festigkeit und
Duktilität sind wie so oft auf ihre Gitterstruktur und atomare Zusammensetzung zurück zu führen. Makroskopische Verformungen
entsprechen mikroskopischen Verschiebungen der Atome gegeneiner und/oder Änderungen ihrer relativen Distanzen zueinander.
Verschiedene Zusammensetzung und Gitterstrukturen lassen diese Verschiebungen mehr oder weniger leicht zu.
Untermauert wurde diese Aussage in Folien 102-105 mit super Abbildungen und der Zusammenhang der Verschiebbarkeit mit
den Netzebenen klingt auch sehr plausibel, allerdings werde ich darüber noch eine Weile meditieren müssen.

Gehab dich wohl, Logbuch.