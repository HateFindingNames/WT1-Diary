\newcommand{\energie}{
    \begin{wrapfigure}{r}{0.3\textwidth}
        \centering
        \includegraphics[width=0.3\textwidth]{./entries/3/energie.png}
        \caption{Gespeicherte Energie bei Erhöhung der Temperatur und Änderung des Phasenzustandes.}
        \label{fig:energie}
    \end{wrapfigure}
}

\newcommand{\wasser}{
    \begin{wrapfigure}{l}{0.4\textwidth}
        \centering
        \includegraphics[width=0.4\textwidth]{./entries/3/wasser.png}
        \caption{Phasenübergangsenergie für Wasser. Quelle: https://homepage.univie.ac.at/ franz.embacher/Splitter/VomEisZumWasserdampf/}
        \label{fig:wasser}
    \end{wrapfigure}
}
Previously on Dennis Hunters log book:
\par\medskip
\energie{}Auf unserer Reise durch das Land der Werkstofftechnik haben wir kristalline Idealstrukturen besichtigt und gelangten von dort aus zu den wirklichkeitsnäheren
Realstrukturen wie wir sie etwa in Legierungen finden. Wie war das noch gleich? Ideale Kristalle können mannigfaltig mit Defekten versehen sein. \textit{Defekt} darf hier allerdings nicht
notwendigerweise negativ konnotiert verstanden werden. Der Begriff bezieht sich auf Unregelmäßigkeiten in der Kristallstruktur manifestiert durch Leerstellen im
Kristallgitter, sich statistisch nicht ausmittelnde Dichteverschiebungen, gegeneinander verschobene Kristallebenen aber auch im Falle von Legierungen in Zwischenräumen
des Kristallgitters eingelagerte Fremdatome.
\par
Defekte bringen auf makroskopischer Ebene gewollte wie ungewollte Effekte mit sich. Die Kunst besteht nun darin das Grundmaterial, wie die einzulagernden Fremdmaterialien
geschickt zu wählen und darüber hinaus sich die physikalischen Zusammenhänge der Kristallisation so zu Nutze zu machen, das gewünschte Eigenschaften gezielt befördert
während unerwünschte möglichst unterdrückt werden.\wasser{}
\par\medskip
Meanwhile:\par\medskip
Recht offensichtlich ist es knifflig Fremdatome in einem Feststoff zu platzieren. Einfacher wird es - und so wird es auch praktiziert - die Materialien zu verflüssigen, zu mischen
und wieder erstarren zu lassen. So drücke ich keine Schokoladenstücke in den gebackenen Teig um die Eigenschaft \textit{Köstlichkeit} eines Brownies gezielt zu manipulieren,
sondern gebe sie in einem bestimmten Verhältnis in die Schmelze (hier der angerührte Teig) und backe sie für eine bestimmte Dauer bei bestimmter Temperatur. Hier beginnt die Analogie
aber auch zu hinken: in der Herstellung bestimmter Werkstoffe ist die Eigenschaft \textit{Köstlichkeit} meist zweitrangig (zumal hier die Schmelze zum Erstarren gekühlt statt erhitzt wird).
\par\medskip
Wie bereits angedeutet können Materialeigenschaften auch durch Manipulation der kristallinen Struktur und der Anordnung ihrer Bezirke eingestellt werden. Hier hilft Kenntnis
über den Mechanismus der Kristallisation aus der Schmelze. Aus der VL haben wir gelernt, dass im Phasenzustand selbst bereits beträchtlich Energie gespeichert ist.
Abbildung \ref{fig:energie} zeigt recht eindrucksvoll, dass beim Übergang in eine andere Phase der Kurvenverlauf nahezu sprunghaft verläuft. Das deckt sich auch gut mit Beobachtungen
in der echten Welt. Wenn ich einen Eiswürfel von nahe 0 \celsius\ in eine Pfütze von nahe 0 \celsius\ verwandele muss ich annähernd genauso viel Energie aufbringen, wie beim Erhitzen
dieser Pfütze auf 100 \celsius. Das wird noch viel dramatischer beim Übergang von flüssig zu gasförmig. Anders herum funktionieren nach dem gleichen Prinzip auch Handwärmer - diese
mit Flüssigkeit gefüllten Kunststoffbeutel die schon so manchen Winterabend versüßt haben. Tatsächlich hielt ich das vor gar nicht langer Zeit noch für schwarze Magie doch Weisheit
vertreibt noch jeden Dämon: In festem Zustand kann ein Handwärmer in kochendem Wasser so lange erhitzt werden, bis sein Inhalt schmilzt. Lasse ich ihn wieder auf Raumtemperatur
herunterkühlen bleibt der Inhalt aber in flüssigem Zustand. Was ist hier los? Es fehlen in der Flüssigkeit kristallisationskeime. Auch die Innenwand des umgebenden Kunststoffbeutels
bietet keine Gelegenheiten um Kristallisationskeime auszubilden. Erst wenn ich das kleine Metallplättchen, dass sich in der Flüssigkeit befindet knicke, kristallisiert die Flüssigkeit
schlagartig von dem Metallplättchen ausgehend aus und gibt dabei deutlich Energie in Form von Wärme frei. Als Kristallisationskeim dient hier die Schallwelle die sich vom Metallplättchen
durch die Flüssigkeit ausbreitet und dort für lokale Dichteunterschiede sorgt.\par
Hierbei wäre es interessant mal zu messen, wie sich das bei unterkühltem Wasser verhält. Kennt man ja auch. Im Sommer eine Flasche Wasser im Gefrierfach vergessen. Sobald man sie öffnet ärgert
man sich erstaunt über das schlagartige Erstarren des Wassers. Wird der Inhalt hierbei wärmer? Memo an mich selbst: am Wochenende die Wärmebildkamera auspacken und rausfinden!

Btw: erkenne ich hier Schnittmengen zu Thermodynamik?!