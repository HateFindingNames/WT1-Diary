In unsrer letzten Session soll es um Wärmebehandlung von Stählen gehen. Aber warum möchte jemand das tun? Und wie?
Wie so oft geht es hier um die gezielte manipulation bestimmter Eigenschaften des Stahls. Wir hatten ja bereits besprochen,
dass das ganz gut durch dosierte Zugabe von Kohlenstoff zum Eisen funktioniert. Weiter lassen sich durch Temperaturprofile
die Gefügestrukturen und -zusammensetzungen in gewünschte Richtungen lenken. So, wie ich das nun verstanden habe geht es
hier nun um die Situation nach der Halbzeugherstellung oder in anderen Worten - die nochmalige Änderung der vorliegenden
Materialeigenschaften durch gezielte Wärmebehandlung.

Welche Arten der Wärmebehandlung gibt es (ohne Anspruch der Vollständigkeit)?

Glühen:
Das Verfahren des Glühens lässt sich in drei grundlegende Schritte einteilen
\begin{enumerate}
    \item Erwärmen zur Solltemperatur
    \item Halten der Solltemperatur
    \item Abkühlen auf Raumtemperatur
\end{enumerate}
Je nach Ziel des des Glühens gibt es hier nochmals verschiedene Bezeichnungen mit jeweils eigenen Temperaturprofilen.

{
\renewcommand{\arraystretch}{1.3}
\begin{table}[h]
    \begin{adjustbox}{width=1\textwidth}
        \begin{tabular}{@{}llll@{}}
            \toprule
            \textbf{Glühen} &
            \textbf{Zweck} &
            \textbf{Temperaturprofil} &
            \textbf{Phys. Hintergrund} \\ \midrule
            Normalglühen &
             Steigerung der Festigkeit und Zähigkeit &
             \begin{tabular}[c]{@{}l@{}}\(20-50 \SI[]{}{\celsius}\) oberhalb der GS-Linie\\ langsames Abkühlen\end{tabular} &
             \begin{tabular}[c]{@{}l@{}}Verkleinerung der Korngröße\\ durch zweimalige Umwandlung\\ des Gefüges \(\alpha \rightarrow \gamma \rightarrow \alpha\)\end{tabular} \\
            Diffusionsglühen &
             \begin{tabular}[c]{@{}l@{}}Homogenere Verteilung der Fremdatome bspw.\\ zur Steigerung der Korrosionsfestigkeit\end{tabular} &
             \begin{tabular}[c]{@{}l@{}}\(1050-1300 \SI[]{}{\celsius}\)\\ Haltezeit bis zu 50h\\ langsames Abkühlen\end{tabular} &
             \begin{tabular}[c]{@{}l@{}}Konzentrationsungleichgewichte sich können durch\\ hohe Temperatur und lange Haltezeit\\ hin zum thermischen Gleichgewicht verschieben\end{tabular} \\
            Spannungsarmglühen &
             \begin{tabular}[c]{@{}l@{}}Entfernen von durch bearbeitende Prozesse\\ wie Tiefziehen verursachten inneren Spannungen\\ unter Beibehalt der sonstigen Eigenschaften\end{tabular} &
             \begin{tabular}[c]{@{}l@{}}\(550-650 \SI[]{}{\celsius}\)\\ ganz langsames Abkühlen\end{tabular} &
             Ausheilen von Versetzungen \\
            Weichglühen &
             Reduktion der Härte \(\rightarrow\) bessere Verformbarkeit &
             \begin{tabular}[c]{@{}l@{}}\(680-780 \SI[]{}{\celsius}\)\\ langsames Abkühlen\end{tabular} &
             Reduktion streifenförmiger Zementitkörner \\ \bottomrule
        \end{tabular}
    \end{adjustbox}
\end{table}
}

Der Spezialfall Härten kann im Grunde oben in das Normalglühen sortiert werden, untergliedert sich aber nochmal durch die
angewandte Abkühlrate und das dadurch bedingte Verhältnis von Martensit zu Ferrit/Perlit für untereutektische bzw. Martensit
und Zementit/Perlit für übereutektische Stähle. Je höher die Abkühlrate desto härter der Stahl bis hin zum \textit{glasharten}
Stahl. Das klassische härten findet durch Abschrecken in Wasser statt und muss anschließend ein weiteres mal Wärmebehandelt
(angelassen) werden um technisch einsetzbar sein zu können. In der realität wird oft bloß die Oberfläche gehärtet
während ein vergleichsweise duktiler Kern über bleibt. So soll das Werkstück verschleißarm werden bei gleichzeitig hoher
Bruchfestigkeit.\newpage

Wie wird in der Praxis Härte gemessen?
Es haben sich gemäß Vorlesung die genormten Härteprüfverfahren nach
\begin{itemize}
    \item Brinell
    \item Vickers
    \item und Rockwell
\end{itemize}
etabliert.\par\medskip

In allen drei Fällen wird ein genormter Prüfkörper unter genormten Bedingungen in den Prüfling eingedrückt und im Anschluss
der zurückgebliebene Abdruck gemessen.

{
\renewcommand{\arraystretch}{1.3}
\begin{table}[h]
    \begin{adjustbox}{width=\textwidth}
        \begin{tabular}{@{}llll@{}}
            \toprule
            \textbf{Prüfverfahren}  & \textbf{Prüfkörper}                                                                                                           & \textbf{Messgröße}                                    & \textbf{Gleichung}                                    \\ \midrule
            Brinell                 & \begin{tabular}[c]{@{}l@{}}Kugel aus\\ gehärtetem Stahl\\ \(D=1-10 \SI{}{mm}\)\\\end{tabular}                                   & Durchmesser \(d\) des Eindrucks                       & \(HB=0,102\cdot\frac{F}{A}\)                               \\
            Vickers                 & \begin{tabular}[c]{@{}l@{}}Diamantpyramide mit\\ Spitzenwinkel \(\varphi=\SI{136}{\degree}\)\\\end{tabular}                     & Diagonale \(d\) des Eindrucks entlang der Oberfläche  & \(HV=0,189\cdot\frac{F}{d^2}\)                             \\
            Rockwell C              & \begin{tabular}[c]{@{}l@{}}Diamantkegel Spitzenwinkel \(\varphi=\SI{120}{\degree}\)\\\end{tabular}                              & Eindrucktiefe \(t_b\)                                 & \(HRC=100-\frac{\SI{0,2}{mm}-t_b}{\SI{0,002}{mm}}\)   \\
            Rockwell B              & Stahlkugel \(D=\SI{0,0625}{mm}\)                                                                                              & Eindrucktiefe \(t_b\)                                 & \(HRB=130-\frac{\SI{0,26}{mm}-t_b}{\SI{0,002}{mm}}\)  \\ \bottomrule
        \end{tabular}
    \end{adjustbox}
\end{table}
}

Erwähnenswert ist, dass jeweils die Gleichungen für Brinell wie für Vickers einen Faktor \(0,102\) beinhalten. Dieser
stammt aus der mittlerweile sogar gesetzlich untersagten Einheit des \textit{Kiloponds} welches definiert war durch
die Kraft, die die sonderbar gewählte Schwerebeschleunigung an der Erdoberfläche auf die Masse eines Kilogramms ausübt
oder anders gesagt
\begin{equation} 
    \SI{1}{kp} = g_N \cdot \SI{1}{kg} = \SI{9,80665}{N}
\end{equation}
Die Härtemaße waren wohl ursprünglich auf das Kilopond bezogen. Der scheinbar vom Himmel gefallene Faktor ist also ein
Umrechnungsfaktor von Kilopond zu Newton mit dem Kehrwert der Schwerebeschleunigung gemäß
\begin{equation}
    \SI{1}{N} = \SI{1}{kp} \cdot 0,102
\end{equation}